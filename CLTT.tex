\documentclass{report}
\usepackage{bbm}
\usepackage{setspace}
\usepackage{tikz}
\usetikzlibrary{chains,fit,shapes,automata,arrows.meta,positioning}
\usepackage{tikz-cd}
\usepackage{amsmath}
\usepackage{stmaryrd}
\usepackage{amsfonts}
\usepackage{amssymb}
\usepackage{enumitem}

\input{src/preamble.tex}

\DeclareMathSymbol{\shortminus}{\mathbin}{AMSa}{"39}
\newcommand{\bE}[0]{\mathbb E}
\newcommand{\bB}[0]{\mathbb B}
\newcommand{\bC}[0]{\mathbb C}
\newcommand{\bD}[0]{\mathbb D}
\newcommand{\bN}[0]{\mathbb N}
\newcommand{\fib}[3]{\begin{array}{l}#1 \\[-0.1cm] \downarrow #2 \\ #3\end{array}}
\newcommand{\fibcent}[3]{\begin{array}{c}#1 \\[-0.1cm] \;\;\downarrow #2 \\ #3\end{array}}
\newcommand{\fibcentt}[2]{\begin{array}{c}#1 \\[-0.1cm] \downarrow \\ #2\end{array}}
\newcommand{\fibPEB}[0]{\fib{\bE}{p}{\bB}}
\newcommand{\Cart}[0]{\mathrm{Cart}}
\newcommand{\idd}[0]{\mathrm{id}}
\renewcommand{\dom}[0]{\mathrm{dom}}
\newcommand{\cod}[0]{\mathrm{cod}}
\newcommand{\Cat}[0]{\mathbf{Cat}}
\newcommand{\Sets}[0]{\mathbf{Sets}}
\newcommand{\PER}[0]{\mathbf{PER}}
\newcommand{\omgSets}[0]{\omega\!\!\shortminus\!\!\Sets}
\newcommand{\Fam}[0]{\mathrm{Fam}}
\newcommand{\Eq}[0]{\mathrm{Eq}}
\newcommand{\reflective}[0]{\begin{array}{c} \leftarrow \\[-0.25cm] \hookrightarrow\\[-0.1cm]\end{array}}
\newcommand{\rs}[0]{\mathrm{s}}
\newcommand{\dbsl}[0]{/\!\!/}
\newcommand{\Sub}[0]{\mathrm{Sub}}
\newcommand{\Rel}[0]{\mathrm{Rel}}
\newcommand{\Per}[0]{\mathrm{Per}}
\newcommand{\ERel}[0]{\mathrm{ERel}}

\newsavebox{\pullbacksq}
\sbox\pullbacksq{%
\begin{tikzpicture}%
\draw (0,0) -- (3ex,0ex);%
\draw (3ex,0ex) -- (3ex,3ex);%
\end{tikzpicture}}

\newcommand{\ul}{\underline}
\renewcommand{\geq}{\geqslant}
\renewcommand{\leq}{\leqslant}

\title{Categorical Logic and Type Theory - Exercises}

\begin{document}

\hypersetup{pageanchor=false}
\begin{titlepage}
\pagestyle{empty}
\makeatletter
\begin{center}

    \vspace*{\fill}

    \rule{\linewidth}{0.5mm} \\[0.4cm]
    {\huge \bfseries \@title}\\[0.1cm]
    \rule{\linewidth}{0.5mm} \\[0.4cm]

    \begin{center} \large
        Killian \textsc{Barbé}\\
        Titouan \textsc{Leclercq}
    \end{center}
    \vspace*{\fill}
\end{center}
\makeatother
\end{titlepage}

\newpage
\hypersetup{pageanchor=true}
\pagestyle{main}
\setcounter{page}{1}

\section{Introduction to fibred category theory}

\subsection{Fibrations}

\begin{exo}[]
    \begin{enumerate}[label=(\roman*)]
        \item Prove the following proposition:

        Cartesian liftings are unique up-to-isomorphism (in a slice): if $f$ and $f'$ with $\cod f =
        \cod f'$ are both Cartesian over the same map, then there is a unique vertical isomorphism
        $\varphi : X \overset{\cong}{\to} X'$ with $f'\circ\varphi = f$.
        \item Suppose $f$ is Cartesian and $g$ and $h$ are above the same map. Show that $f\circ g = f
        \circ h$ implies $g = h$.
    \end{enumerate}
\end{exo}

\begin{answer}
    \begin{enumerate}[label=(\roman*)]
        \item Let $f : X \to Y$ and $f' : X' \to Y$ in $\bE$ be both Cartesian over a map $u : pX \to
        pY$. Then as $f$ is Cartesian, there is a unique $\psi : X' \to X$ such that $f' =\psi \circ 
        f$ and as $f'$ is cartesian, there is a unique $\varphi : X \to X'$ such that $f = \varphi
        \circ f'$. We claim that $\psi = \varphi ^{-1}$. Indeed, the composite $\varphi\circ \psi$
        is such that $f'\circ (\varphi\circ\psi) = f' = f'\circ \idd$, so by uniqueness of the arrow
        $h$ such that $f' = f'\circ h$, we deduce that $\varphi\circ\psi = \idd$. Similarly, for
        $\psi\circ\varphi$, we have that $f\circ(\psi\circ\varphi) = f$, thus $\psi\circ\varphi =
        \idd$. This means that $\varphi : X \overset{\cong}{\to} X'$ is such that $f'\circ \varphi = f$
        and is unique by uniqueness of the definition of $\varphi$. The fact that $\varphi$ is vertical
        is because we used Cartesianness of $f$ (resp $f'$) on the functions $u,u,\idd$, so
        $p(\varphi) = \idd$.
        \begin{center}
            \begin{tikzcd}
                \bE\ar[ddd,"p"] & & X \ar[dd,bend left,dashed,"\varphi"] \ar[rrd,"f"] & & \\
                & & & & Y\\
                & & X' \ar[uu,bend left, dashed, "\psi"] \ar[rru, "f'"] & \\
                \bB & & I \ar[rr,"u"] & & J
            \end{tikzcd}
        \end{center}
        \item Let $f : X \to Y$ be Cartesian and $g,h : Z \to X$ above the same map. Suppose that
        $f\circ g = f\circ h$.
        Then, if we write $t = f\circ g$ (which is also $f\circ h$), we have that $g$ and $h$ both
        complete the triangle below:
        \begin{center}
            \begin{tikzcd}
                \bE\ar[ddd,"p"] & & & X \ar[r,"f"] & Y\\
                & & Z\ar[ur,"g",shift left = 1ex]\ar[ur,"h" below,shift right = 1ex] 
                \ar[urr,bend right,"t" below]\\
                & & & J \ar[r,"u"]& I \\
                \bB & & L\ar[ur,"w"]\ar[urr,"v" below, bend right] & & 
            \end{tikzcd}
        \end{center}
        But $f$ being cartesian, there can be only one such arrow, so $g = h$.
    \end{enumerate}
\end{answer}

\begin{exo}
    Let $p : \bE\to\bB$ be a functor; assume $f : X \to Y$ is in $\bE$ and put $u = pf$. Show that $f$
    is Cartesian if and only if for each $Z \in \bE$ and $v : pZ\to pX$ in $\bB$, the function
    \begin{center}
        \begin{tikzcd}
            \bE_v\big(Z,X\big) \ar[rr,"f\;\circ\;(-)"] & & \bE_{u\circ v} \big(Z,Y\big)
        \end{tikzcd}
    \end{center}
    is an isomorphism.
\end{exo}

\begin{answer}
    Suppose that for all $Z\in \bE$ and $v : pZ\to pX$ in $\bB$, $f\circ(-)$ is an isomorphism between
    $\bE_v(Z,X)$ and $\bE_{u\circ v}(Z,Y)$, let's show that $f$ is Cartesian. Let $g : Z \to X$ be an arrow in
    $\bE$, and $w : pZ \to pX$ be an arrow in $\bB$ such that $pg = pf\circ w$. The fact that $f$ is Cartesian
    means that there is one and only one arrow $h : Z \to X$ above $w$ such that $g = f\circ h$, which is
    exactly the fact that $f\circ(-)$ is a bijection from $\bE_{w}(Z,X)$ to $\bE_{u\circ w}(Z,Y)$.

    Conversely, if $f$ is Cartesian, let $Z \in \bE$ and $v : pZ \to pX$ in $\bB$. Let $g \in \bE_{u\circ v}
    (Z,Y)$, the only $h\in \bE_v(Z,X)$ such that $f\circ h = g$ is exactly the one given by the universal
    property of $f$ being Cartesian.
\end{answer}

\begin{exo}
    Consider the total category of a fibration. Show that
    \begin{enumerate}[label=(\roman*)]
        \item every morphism factors as a vertical map followed (diagrammatically) by a Cartesian one;
        \item\label{1.1.1.3.ii} a Cartesian map above an isomorphism is an isomorphism. Especially a
        vertical Cartesian map is an isomorphism.
    \end{enumerate}
\end{exo}

\begin{answer}
    \begin{enumerate}[label=(\roman*)]
        \item Let $X,Y\in\bE$ and $f : X \to Y$. As $pf : pX\to pY$ and $p$ is a fibration, we can lift
        $pf$ in a function $h : Z \to Y$ which is Cartesian. As $h$ is Cartesian and $pf = pf\circ\idd$,
        there exists one and only one function $g : Z \to X$ above $\idd$ such that $f = h \circ g$. Thus,
        $f$ factors as a vertical map followed by a Cartesian one.
        \item Let $f : X \to Y$ be above an isomorphism $u : I \to J$ and be Cartesian. As $u$ is an
        isomorphism, let $v = u^{-1} : J \to I$. We know that $\idd = u\circ v$ so, as $f$ is Cartesian, there
        exists one and only one $g : Y \to X$ above $u$ such that $\idd = f \circ g$. We want to show that
        $g\circ f = \idd$, but as $f\circ (g\circ f) = f = f \circ \idd$, and both $\idd$ and $g\circ f$ are
        vertical, this means by uniqueness that $g\circ f = \idd$, so $g =f^{-1}$.
        \begin{center}
            \begin{tikzcd}
                \bE\ar[ddd,"p"] & & X\ar[rr,bend left, "f"]& & Y\ar[ll,dashed,bend left,"g"] \\
                & & & & \\
                & & & & \\
                \bB & & I \ar[rr, bend left, "u"]& & \ar[ll,bend left, "v"]J
            \end{tikzcd}
        \end{center}
    \end{enumerate}
\end{answer}

\begin{exo}
    Let $\fibPEB$ be a fibration. Prove that
    \begin{enumerate}[label=(\roman*)]
        \item all isomorphisms in $\bE$ are Cartesian;
        \item\label{1.1.1.4.ii} if $\overset{f}{\rightarrow}\overset{g}{\rightarrow}$ is a composable
        pair of Cartesian morphisms in $\bE$, then also their composite $\overset{g\circ f}{\rightarrow}$ is Cartesian.
    \end{enumerate}
    Hence it makes sense to talk about the subcategory $\Cart(\bE)\hookrightarrow\bE$ having
    all objects from $\bE$ but only the Cartesian arrows. We write $|p|$ for the composite
    $\Cart(\bE)\hookrightarrow\bE \rightarrow \bB$.
    \setlist[enumerate]{resume}
    \begin{enumerate}[label=(\roman*)]
        \item\label{1.1.1.4.iii} Let $\overset{f}{\to}\overset{g}{\to}$ be a composable pair in $\bE$
        again. Show now that if $g$ and $g\circ f$ are Cartesian, then $f$ is Cartesian as well.
        \item Verify that a consequence of \ref{1.1.1.4.iii} is that the functor $|p| : \Cart(\bE) \to
        \bB$ is a fibration. From \ref{1.1.1.3.ii} in the previous exercise it follows that all
        fibres of $|p|$ are groupoids (\textit{i.e.} that all maps in the fibres are isomorphisms).
    \end{enumerate}
    [This $|p|$ will be called the \textbf{fibration of objects of} $p$.]
\end{exo}

\begin{answer}
    \begin{enumerate}[label=(\roman*)]
        \item This directly follows from the characterization of $f$ being Cartesian if and only if $(-)\circ
        f$ is an isomorphism between functions over given arrows.
        \item Let $\xrightarrow f \xrightarrow g$ be a composable pair of Cartesian morphisms in $\bE$.
        Let's show that $g\circ f$ is Cartesian. We state $f : X \to Y$ and $g : Y \to Z$. Let
        $h : A \to Z$ and $w : pA \to pX$ such that $ph = pg\circ pf \circ w$, let's show that there exists
        one and only one $k : A \to X$ such that $h = g \circ f \circ k$. First, using the fact that $g$ is
        Cartesian on $h$ and $pf\circ w$, there is only one $h' : A \to Y$ such that $ph' = pf \circ w$ and
        $g\circ h' = h$. Know, using the fact that $f$ is Cartesian with $ph' = pf\circ w$, we find that there
        exists one and only one $k : A \to X$ such that $pk = w$ and $h' = f\circ k$, so there exists one and
        only one $k$ such that $g\circ h' = g\circ f \circ k$, \textit{i.e.} such that $h = g\circ g \circ k$.
        So $g\circ f$ is Cartesian.
        \begin{center}
            \begin{tikzcd}
                \bE\ar[dddd,"p"] & & A\ar[dr,dashed,"k"] \ar[drr,"h'" near end, dashed, bend left = 10]
                \ar[drrr,"h",bend left=15] & & &\\
                & & & X\ar[r,"f"] & Y\ar[r,"g"] & Z \\
                \\
                & & pA\ar[dr,"w"] \ar[drrr,"ph", bend left=15]\\
                \bB & & & pX\ar[r,"pf"] & pY\ar[r,"pg"] & pZ
            \end{tikzcd}
        \end{center}
        \item Suppose that $g$ and $g\circ f$ are Cartesian, let's prove that $f$ is Cartesian. Suppose that
        $f : X \to Y$, $g : Y \to Z$ and that we have $h : A \to Y$ and $w : pA \to pX$ such that
        $ph = pf \circ w$, let's prove that there exists a unique $k : X \to A$ such that $pk = w$ and
        $h = f\circ k$. First, as $g$ is Cartesian, $h$ is the only arrow above $ph$ such that $g\circ h =
        g\circ h$. Thus the condition on $k$ is equivalent to the condition that $g\circ f\circ k = g\circ h$
        and that $k$ is above $w$. But then, we use the fact that $g\circ f$ is Cartesian to deduce that
        there is only one such $k$. Thus, $f$ is Cartesian.
        \begin{center}
            \begin{tikzcd}
                \bE\ar[dddd,"p"] & & A\ar[dr,dashed,"k"] \ar[drr,"h" near end, bend left = 10]
                \ar[drrr,"g\circ h", bend left = 15]& & &\\
                & & & X\ar[r,"f"] & Y\ar[r,"g"] & Z \\
                \\
                & & pA\ar[dr,"w"] \ar[drr,"ph" near end, bend left=10]\ar[drrr,"p(g\circ h)", bend left=15]\\
                \bB & & & pX\ar[r,"pf"] & pY\ar[r,"pg"] & pZ
            \end{tikzcd}
        \end{center}
        \item Let's check that $|p|$ is indeed a fibration. Let $X \in \bE$, $I\in\bB$ and $u : I \to pX$.
        As $p$ is a fibration, we can find a Cartesian morphism $f : Y \to X$ in $\bE$ such that $pf = u$.
        By definition, $f \in \Cart(\bE)$, so we just need to check that $f$ is Cartesian inside $\Cart(\bE)$,
        but this is straightforward. All fibres of $|p|$ being groupoids is due to the fact that they are
        Cartesian morphisms above $\idd$, which is an isomorphism, and hence are themselves isomorphisms.
    \end{enumerate}
\end{answer}

\begin{exo}
    Verify that the following two results---known as the Pullback Lemmas--- are a consequence of 
    \ref{1.1.1.4.ii} and \ref{1.1.1.4.iii} from the previous exercise. Consider
    \begin{center}
        \begin{tikzcd}
            \bullet \ar[rr] \ar[dd] & & \bullet \ar[rr]\ar[dd] & & \bullet\ar[dd]\\
            & (B) & & (A) & \\
            \bullet \ar[rr] & & \bullet\ar[rr] & & \bullet
        \end{tikzcd}
    \end{center}
    \begin{enumerate}
        \item If $(A)$ and $(B)$ are pullback squares, then the outer rectangle is also a pullback
        square.
        \item If the outer rectangle and $(A)$ are pullback squares, then $(B)$ is a pullback square
        as well.
    \end{enumerate}
\end{exo}

\begin{exo}
    Consider a functor $p : \bE \to \bB$. We describe a slightly weaker notion of Cartesianness, than
    the one above. Call a morphism $f : X \to Y$ in $\bE$ \textbf{weak Cartesian} if for each
    $g : Z \to Y$ with $pf = pg$  there is a unique vertical $g : Z \to X$ in with $f\circ h = g$.
    Show that the functor $p$ is a fibration if and only if both
    \begin{enumerate}[label=(\alph*)]
        \item every morphism $u : I \to pY$ in $\bB$ has a weak Cartesian lifting $f : X \to Y$;
        \item the composition of two weak Cartesian morphisms is again weak Cartesian.
    \end{enumerate}
\end{exo}

\begin{exo}
    Check that the following are (trivial) examples of fibrations
    $$\fib{\bB\times\bC}{\mathrm{fst}}{\bB}\qquad \fib{\bB}{\idd}{\bB}\qquad \fib{\bB}{}{1 = \{*\}}
    \qquad \fib{X}{}{I}$$
    where $X,I$ are sets (\textit{i.e.} discrete categories).
\end{exo}

\begin{answer}
    \textbf{First case:} $\fib{\bB\times\bC}{\mathrm{fst}}{\bB}$
    Let $\langle X,Y\rangle \in\bB\times \bC$, $Z \in\bB$, and $u : Z\to X$. We claim that the Cartesian
    lifting of $u$ is $\langle u,\idd_Y\rangle : \langle Z,Y\rangle \to \langle X,Y\rangle$. Indeed, if one
    has $\langle h_1,h_2\rangle : \langle A,B\rangle \to \langle X,Y\rangle$ and $w : A \to Z$ such that
    $u\circ w = h_1$, then a $\langle k,\ell\rangle : \langle A,B\rangle \to \langle Z,Y\rangle$ such that
    $\mathrm{fst}(\langle k,\ell\rangle) = w$ means that $k = w$, and having
    $$\langle u,\idd_Y\rangle \circ \langle k,\ell\rangle = \langle h_1,h_2\rangle$$
    means, by projection on the second coordinates, that $\ell = h_2$. So the only candidate (and it obviously
    works) is the arrow $\langle w,h_2\rangle$.
    \begin{center}
        \begin{tikzcd}[ampersand replacement=\&]
            {\bB\times\bC} \&\& {\langle A,B\rangle} \\
            \&\&\& {\langle Z,Y\rangle} \&\& {\langle X,Y\rangle} \\
            \\
            \&\& A \\
            {\bB} \&\&\& Z \&\& X
            \arrow["{\mathrm{fst}}", from=1-1, to=5-1]
            \arrow["{\langle u,\idd_Y\rangle}", from=2-4, to=2-6]
            \arrow["{\langle h_1,h_2\rangle}"' above, from=1-3, to=2-6, bend left = 15]
            \arrow["{\langle k,\ell\rangle}", from=1-3, to=2-4]
            \arrow["k", from=4-3, to=5-4]
            \arrow["{h_1}"' above, from=4-3, to=5-6, bend left = 15]
            \arrow["u", from=5-4, to=5-6]
        \end{tikzcd}
    \end{center}

    \textbf{Second case:} $\fib{\bB}{\idd}{\bB}$ Let $X \in \bB$, $Y \in \bB$ and $u : Y \to X$. Its Cartesian
    lifting is itself: if we have a triangle below that commutes, then it can only be lifted to itself as
    the functor is the identity.

    \textbf{Third case:} $\fib{\bB}{}{\{*\}}$ Let $X \in \bB$, $* \in 1$. The only arrow in $1$ is the
    identity of the sole object, and we can take as Cartesian lifting of $\idd_*$ the arrow $\idd_X$. The
    fact that it is Cartesian is because it is an isomorphism.

    \textbf{Fourth case:} $\fib{X}{}{I}$ where $X$ and $I$ are discrete categories. Here too, the only
    possible arrows are identities, which can be lifted directly be an identity on the above object, which
    is Cartesian because it is an isomorphism.
\end{answer}

\begin{exo}
    For an arbitrary category $\bB$, consider the domain functor $\dom : \bB^\rightarrow \to \bB$.
    \begin{enumerate}[label=(\roman*)]
        \item Describe the fibre category above $I\in\bB$. It is usually called the \textbf{opslice
        category} or simply \textbf{opslice} and written as $I\backslash\bB$.
        \item Show that $\dom$ is a fibration (without any assumptions about $\bB$).
        \item Show also that for each $I\in\bB$ the domain functor $\dom_I : \bB/I\to\bB$ is a
        fibration.
    \end{enumerate}
\end{exo}

\begin{answer}
    \begin{enumerate}[label=(\roman*)]
        \item The fibre category $I\backslash \bB$ has :
        \begin{itemize}
            \item as objects, pairs $\langle X,f\rangle$ where $f : I \to X$.
            \item as morphisms $\langle X,f\rangle \xrightarrow{h} \langle Y,g\rangle$ an arrow $h : X \to Y$
            such that the following diagram commutes:
            \begin{center}
                \begin{tikzcd}
                    & I \ar[ddl,"f" left] \ar[ddr,"g"] \\
                    \\
                    X\ar[rr,"h"] & & Y
                \end{tikzcd}
            \end{center}
            \item Let $f : X \to Y$ an element of $\bB^\to$, $Z \in \bB$ and $u : Z \to X$. We claim that a
            Cartesian lifting of $u$ is given by
            $$\langle u , \idd_Y\rangle : \fib{Z}{f\circ u}{Y}\longrightarrow\fib{X}{f}{Y}$$
            To show that it is Cartesian, let $\langle h_1,h_2\rangle : \fib{A}{\alpha}{B} \longrightarrow
            \fib{X}{f}{Y}$ and $\beta : A \to Z$ such that $h_1 = u\circ \beta$. Let's show that the only
            arrow $\langle t,u\rangle : \fib{A}{\alpha}{B}\longrightarrow\fib{X}{f}{Y}$ above $\beta$ such
            that $\langle h_1,h_2\rangle = \langle u,\idd_Y\rangle \circ \langle t,u\rangle$ is
            $\langle \beta,h_2\rangle$. This is direct, as being above $\beta$ means that $t = \beta$, and
            the other equality amounts to $h_2 = \beta$. Thus this lifting is Cartesian. This shows that
            $\dom : \bB^\to\to\bB$ is a fibration.
            \item Let's show that $\dom_I$ is a fibration. Let $\fib{X}{f}{I} \in \bB/I$, $Y\in \bB$ and
            $u : Y \to X$. A Cartesian lifting is given by
            $$u : \fib{Y}{f\circ u}{I}\longrightarrow \fib{X}{f}{I}$$
            and the proof that this lifting is Cartesian is the same as in the previous question.
        \end{itemize}
    \end{enumerate}
\end{answer}

\begin{exo}
    Assume $\bB$ is a category with pullbacks. Show that the functor $\bB^{\to\to}\to\bB^\to$ sending
    $\overset{f}{\to}\overset{g}{\to}$ to $\overset{g}{\to}$ is a fibration. Is the composite
    $\bB^{\to\to}\to\bB^\to\overset{\cod}{\to}\bB$ also a fibration?
\end{exo}

\begin{exo}
    Show that the object functor $\Cat\to\Sets$ is a fibration. Also that the forgetful functor
    $\mathbf{Sp}\to\Sets$ is a fibration --- where $\mathbf{Sp}$ is the category of topological spaces
    and continuous functions.
\end{exo}

\begin{exo}
    Let $\mathbf{Fld}$ be the category of fields and field homomorphisms; $\mathbf{Vect}$ is the
    category of vector spaces: objects are triples $(K,V,\cdot)$ where $K$ is a field of scalars, $V$
    is an Abelian group of vectors and $\cdot : K\times V \to V$ is an action of scalar multiplication
    (which distributes both over scalar and over vector addition). A morphism $(K,V,\cdot)\to
    (L,W,\cdot)$ in $\mathbf{Vect}$ is a pair $(u,f)$ where $u : K \to L$ is a field homomorphism and
    $f : V \to W$ a group homomorphism such that $f(a\cdot x) = u(a)\cdot f(x)$ for all $a\in K$ and
    $x \in V$.\\
    Check that the obvious forgetful functor $\mathbf{Vect}\to\mathbf{Fld}$ is a fibration. What are
    the fibres? Which maps are Cartesian?
\end{exo}

\subsection{Some concrete examples: sets, \texorpdfstring{$\omega$}{omega}-sets and PERs}

\begin{exo}
    Prove that a morphism $(u,(f_i)_{i\in I})$ in $\Fam(\bC)$ is Cartesian if and only if each $f_i$ is an
    isomorphism in $\bC$.
\end{exo}

\begin{exo}
    Consider a map $f = (f_i : X_i\to Y_i)_{i\in I}$ in the fibre $\Fam(\bC)_I = \bC^I$ over $I\in\Sets$.
    Prove that $f$ is a mono in this fibre if and only if each $f_i$ is a mono in $\bC$.
\end{exo}

\begin{exo}
    For an arbitrary category $\bB$, let $\bB^\to_\bullet$ be the category with \textbf{pointed families}
    as objects; these are pairs $\langle\left(\fib{X}{\varphi}{I}\right),s\rangle$ where $s$ is a section of
    $\varphi$ (\textit{i.e.} a map $s : I \to X$ with $\varphi\circ s = \idd$). A morphism $\langle
    \left(\fib{X}{\varphi}{I}\right),s\rangle \longrightarrow \langle \left(\fib{Y}{\psi}{J}\right),t\rangle$
    in $\bB^\to_\bullet$ consists of a pair of morphisms $u : I \to J$, $f : X \to Y$ in $\bB$ with $\psi 
    \circ f = u \circ \varphi$ and also $f \circ s = t \circ u$. Thus morphisms of pointed families preserve
    the points (\textit{i.e.} sections) of the families. Prove that
    \begin{enumerate}[label=(\roman*)]
        \item if the category $\bB$ has pullbacks then the functor $\bB^\to_\bullet \to \bB$ sending
        $\langle\left(\fib{X}{\varphi}{I}\right),s\rangle$ to the index object $I$ is a fibration.
        \item for $\bB = \Sets$, there is an equivalence of categories $\Fam(\Sets_\bullet)
        \overset{\simeq}{\to} \Sets^\to_\bullet$ like in Proposition 1.2.2, where $\Sets_\bullet$ is the
        category of \textbf{pointed sets}: objects are sets containing a distinguished base point, morphisms
        are functions preserving such points.
    \end{enumerate}
\end{exo}

\begin{exo}
    Check that for a PER $R$ one has $R\subseteq |R|^2$.
\end{exo}

\begin{exo}
    Let $I$ be a set. A \textbf{partition} of $I$ is a collection $Q\subseteq P(I)$ of subsets of $I$
    satisfying $(1)$ every set in $Q$ is nonempty $(2)$ if $a,b\in Q$ and $a\cap b \neq \emptyset$, then
    $a=b$ $(3)$ $\bigcup Q = I$. A \textbf{partial partition} of $I$ is a subset $Q\subset P(I)$ satisfying
    $(1)$ and $(2)$ but not necessarily $(3)$. Show that
    \begin{enumerate}[label=(\roman*)]
        \item there is a bijective correspondence between partitions and equivalence relations and between
        partial partitions and partial equivalence relations (on $I$).
        \item there is a bijective correspondence between partial equivalence relations on $I$ and
        equivalence relations on subsets of $I$.
    \end{enumerate}
\end{exo}

\begin{exo}
    Notice that for $R\in \PER$, the ``global sections" of ``global elements" homset $\PER(1,R)$ is
    isomorphic to the quotient $\bN/R$. And also that all homsets in $\PER$ and in $\omgSets$ are
    countable.
\end{exo}

\begin{exo}
    \begin{enumerate}[label=(\roman*)]
        \item Prove that for each $\omega$-set $(I,E)$ the slice category $\omega$-$\Sets/(I,E)$ is
        Cartesian closed, \textit{i.e.} that $\omgSets$ is a locally Cartesian closed category (LCCC).
        \item Show that also $\PER$ is a CCC.
    \end{enumerate}
\end{exo}

\begin{exo}
    Show that a map $\nabla X \to (\bN/R,\in)$ in $\omgSets$ is constant (where $X$ is a set and $R$ is a
    PER).
\end{exo}

\begin{exo}
    \begin{enumerate}[label=(\roman*)]
        \item \label{1.2.9.i} Prove that the unit $\eta_{(X,E)}$ of the reflection
        $\PER\reflective
        \omgSets$ at $(X,E)\in \omgSets$ is an isomorphism if and only if the existence predicate $E : X \to
        P\bN$ has disjoint images (\textit{i.e.} satisfies $E(x)\cap E(y) \neq \emptyset \implies x=y$).\\
        Conclude that $\PER$ is equivalent to the full subcategory of $\omgSets$ on these objects with such
        disjoints images. These $\omega$-sets are also called \textbf{modest sets} (after D. Scott). In this
        situation the existence predicate $E : X \to P\bN$ may equivalently be described via a surjection
        function $U \twoheadrightarrow X$, where $U\subseteq \bN$ (\textit{i.e.} via a subquotient of $\bN$),
        see \textit{e.g.} [ref].
        \item In view of \ref{1.2.9.i}, describe the reflector $\mathbf{r} : \omgSets \to \PER$ as `forcing
        images to be disjoint', by taking a suitable quotient
        $$(X\to P\bN) \mapsto (X/\!\!\sim\; \to P\bN)$$
    \end{enumerate}
\end{exo}

\begin{exo}
    \begin{enumerate}[label=(\roman*)]
        \item Let $\Eq(\bN) = \{(n,n)\mid n\in \bN\}\subseteq \bN\times\bN$ be the `diagonal' PER. Show
        that it is a natural number object (NNO) in $\PER$. Also that the resulting $\omega$-set $N =
        (\bN,\in)$ with $\in (n) = \{n\}$ is NNO in $\omgSets$.
        \item Check that the maps $\Eq(\bN)\to \Eq(\bN)$ in $\PER$, \textit{i.e.} the maps $N\to N$ in
        $\omgSets$, can be identified with the (total) recursive functions $\bN\to\bN$.
    \end{enumerate}
\end{exo}

\begin{exo}
    Show that the category $\omgSets$ has finite colimits. And conclude, from the reflection
    $\PER\reflective \omgSets$ that $\PER$ also has finite colimits.
\end{exo}

\begin{exo}
    Prove that the reflector $\mathbf{r} : \omgSets\to\PER$ preserves finite products, but does \textit{not}
    preserve equalisers.\\
    $[$\textit{Hint.} For a counter example, consider in $\Sets$ on the two element set $2 = \{0,1\}$ the
    identity and twist maps $\idd,\lnot : 2 \rightrightarrows 2$, with empty set $\emptyset\to 2$ as
    equaliser. By applying $\nabla : \Sets \to \omgSets$ we get an equaliser diagram in $\omgSets$ (since
    $\nabla$ is right adjoint). But it is not preserved by the reflector $\mathbf{r}$, since $\mathbf{r}
    (\nabla 2)$ is terminal, and $\mathbf{r}(\nabla\emptyset)$ is initial.$]$
\end{exo}

\begin{exo}
    \begin{enumerate}[label=(\roman*)]
        \item Prove that $\Fam(-) : \Cat \to \Cat$ is a $(2\shortminus)$functor. (One has to ignore aspects
        of size here, because categories $\Fam(\bC)$ are not small; for example, $\Fam(\boldsymbol{1})$ is
        isomorphic to $\Sets$.)
        \item Show that $\Fam(\bC)$ is the free completion of $\bC$ with respect to set-indexed coproducts.
        This means that $\Fam(\bC)$ has set-indexed coproducts and that there is a unit $\bC\to\Fam(\bC)$
        which is universal among functors from $\bC$ to categories $\bD$ with set-indexed coproducts
        $\coprod_{i\in I} X_i$.
        \item Prove that a category $\bC$ has arbitrary coproducts if and only if the unit $\bC\to\Fam(\bC)$
        has a left adjoint.
    \end{enumerate}
    $[$The $\Fam(-)$ operation forms a so-called `KZ-doctrine', see [ref].$]$
\end{exo}

\subsection{Some general examples}

\begin{exo}
    \begin{enumerate}[label=(\roman*)]
        \item Show that in the total category $\rs(\bB)$ of a simple fibration $\fibcent{\rs(\bB)}{\;\;}{\bB}$ a morphism $(u,f) : (I,X) \to (J,Y)$ is Cartesian if and only if there i an isomorphism
        $h : I \times X \overset{\cong}{\to} I \times Y$ in $\bB$ such that $\pi\circ h = \pi$ and
        $\pi'\circ h = f$.
        \item Show that the assignment $(I,X)\mapsto I^*(X) = \left(\fibcent{I\times X}{\pi}{I}\right)$
        extends to a full and faithful functor $\rs(\bB)\to\bB^\to$. Prove that it maps Cartesian morphisms
        to pullback squares.\\
        $[$This functor restricts to a full and faithful functor $\bB \dbsl I\to\bB/I$.$]$
    \end{enumerate}
\end{exo}

\begin{exo}
    Consider a simple fibration $\fibcent{\rs(\bB)}{\;\;}{\bB}$ for a category $\bB$ with finite products
    $(1,\times)$. Prove that
    \begin{enumerate}[label=(\roman*)]
        \item each fibre $\bB\dbsl I$ has finite products, and $I^* : \bB\to\bB\dbsl I$ preserves these
        products;
        \item the following are equivalent:
        \begin{enumerate}[label=(\alph*)]
            \item $\bB$ is Cartesian closed;
            \item each fibre $\bB\dbsl I$ is Cartesian closed;
            \item each functor $I^* : \bB\to \bB\dbsl I$ has a right adjoint $ I \Rightarrow (-)$.
        \end{enumerate}
    \end{enumerate}
\end{exo}

\begin{exo}
    In case a category $\bB$ has finite limits (\textit{i.e.} additionally has equalisers with respect to
    the previous exercise), prove that $\bB$ is Cartesian closed if and only if for each $I \in \bB$, the
    functor $I^* : \bB\to\bB\dbsl I$ (to the ordinary slice) mapping $X$ to $I^*(X) =
    \left(\fibcent{I\times X}{\pi}{I}\right)$, has a right adjoint $\prod_I$.\\
    $[$\textit{Hint.} One obtains a right adjoint $\prod_I$ by mapping the family
    $\left(\fibcent{X}{\varphi}{I}\right)$ to the domain of the equaliser $e$ in
    \begin{center}
        \begin{tikzcd}
            \prod_I(\varphi)\ar[r,"e",tail] & (I\Rightarrow X) \ar[rr,bend left,"I\Rightarrow \varphi"]
            \ar[rr,bend right, "\Lambda(\pi)" below] & & (I\Rightarrow I)
        \end{tikzcd}.
    \end{center}
    \vspace{-1.58cm}
    \begin{flushright}
        $]$
    \end{flushright}
\end{exo}

\vspace{0.5cm}

\begin{exo}
    Let $\bB$ be a category with finite products and $I$ an object of $\bB$.
    \begin{enumerate}[label=(\roman*)]
        \item Show that the functor $I\times (-) : \bB\to\bB$ forms a comonad in $\bB$.
        \item Show that the simple slice $\bB\dbsl I$ is the Kleisli category of this comonad $I\times(-)$ and
        that the ordinary slice $B/I$ is its Eilenberg-Moore category.
    \end{enumerate}
\end{exo}

\begin{exo}
    Let $\bB$ have finite limits. Prove that a map of families $\left(\fibcent{X}{\!\!}{I}\right)\to
    \left(\fibcent{Y}{\!\!}{J}\right)$ is a mono in $\bB^\to$ if and only if both its components $I\to J$ and
    $X \to Y$ are monos in $\bB$.
\end{exo}

\begin{exo}
    A \textbf{regular mono} is a mono that occurs as an equaliser. Write $\mathrm{RegSub}(\bB)$ for the full
    subcategory $\Sub(\bB)$ consisting of (equivalence classes of) regular monos. Show that the codomain
    functor $\fibcent{\mathrm{RegSub}(\bB)}{\!\!}{\bB}$ is a fibration.
\end{exo}

\begin{exo}
    Let $\bB$ be a category with finite limits.
    \begin{enumerate}[label=(\roman*)]
        \item Show that is $\bB$ is a CCC then also $\bB^\to$ is a CCC and $\fib{\bB^\to}{}{\bB}$ is a functor
        which strictly preserves the CCC-structure.
        \item Show that the same holds for $\Sub(\bB)$ instead of $\bB^\to$.
    \end{enumerate}
    $[$\textit{Hint.} For families $\left(\fibcent{X}{\varphi}{I}\right)$ and
    $\left(\fibcent{Y}{\psi}{J}\right)$ construct the exponent family $\varphi\Rightarrow \psi$ over the
    exponent object $I\Rightarrow J$ in $\bB$ as in the following pullback diagram.
    \begin{center}
    \begin{tikzcd}[sep=6em]
        U \ar[dr,phantom, "\usebox\pullbacksq", very near start] \ar[rr] 
        \ar[d, "\varphi\Rightarrow \psi" left]& & (X\Rightarrow Y)\ar[d,"X\Rightarrow \psi"]\\
        (I\Rightarrow J) \ar[rr,"\varphi\Rightarrow J"] & {} & (X\Rightarrow J)
    \end{tikzcd}
    \end{center}
    \vspace{-1.1cm}
    \begin{flushright}
        $]$
    \end{flushright}
\end{exo}

\begin{exo}
    Give a categorical formulation of anti-symmetry of a relation $R\rightarrowtail I\times I$.
\end{exo}

\begin{exo}
    Verify in detail that the following functors are fibrations.
    $$\fibcentt{\Sub(\bB)}{\bB}\qquad\fibcentt{\Rel(\bB)}{\bB}\qquad \fibcentt{\Per(\bB)}{\bB}
    \qquad \fibcentt{\ERel(\bB)}{\bB}$$
\end{exo}

\begin{exo}
    Define an alternative category of relations $R\rightarrowtail I\times J$ on possibly different objects
    in a category $\bB$, which is fibred over $\bB\times\bB$.
\end{exo}

\begin{exo}
    Let $\fibPEB$ be a fibration. Prove that $p$ is preordered (\textit{i.e.} all its fibre categories are
    preorders) if and only if above each map $pX\to pY$ in $\bB$ there is at most one arrow $X\to Y$ in $\bE$
    (\textit{i.e.} if $p$ is faithful). Conclude that in the total category of a preorder fibration, a
    vertical morphism is monic.
\end{exo}

\end{document}